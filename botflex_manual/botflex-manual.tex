% acmtr.tex
% revised 1/20/97
% updated 06/01/01
% $Header: acmtr.tex,v 1.5 2/14/96 11:07:57 boyland Exp $

\documentclass[acmtocl]{acmtrans2m}


\newcommand{\BibTeX}{{\rm B\kern-.05em{\sc i\kern-.025em b}\kern-.08em
    T\kern-.1667em\lower.7ex\hbox{E}\kern-.125emX}}

\markboth{Sheharbano Khattak}{BOTFLEX REFERENCE MANUAL}

\title{BOTFLEX REFERENCE MANUAL}
\author{SHEHARBANO KHATTAK\\SYSNET, NATIONAL UNIVERSITY OF COMPUTER AND EMERGING SCIENCES\\ISLAMABAD, PAKISTAN.
}
\begin{abstract}
This document describes BotFlex, an open source tool for botnet detection. Discussion includes the philosophy behind BotFlex, what it can do, what it cannot do, to do list, and how other developers can extend it. As BotFlex is in its elementary stage, this document will be extended and modified during the course of its refinement.   

\end{abstract}

\begin{document}


%\setcounter{page}{111}


\maketitle


\section{Introduction}
Despite efforts to detect and contain botnets,
the phenomenon continues to plague our cyber infrastructure.
Several reports \textbf{reference} bear testimony to the fact
that botnet operators share information and services to faciliate
their propagation and survivability. 
Unfortunately, the defense landscape lacks the degree
of cooperation required to address the botnet phenomenon.
We have managed to churn hundreds of reports and research papers 
to address the issue of botnet measurement, detection and defense.
However, \textbf{no} open source tool for botnet detection exists.
This is particularly surprising in view of the fact that botnets
have been around for a while.
Some schools of thought may argue that releasing botnet
specific detection and defense tools in open source
will help botnet creators to evade them. 
This approach advocates `security by obscurity' which has been
proved wrong time and again.  
To address the issue highlighted above, we propose to build
an open source bot detection tool, BotFlex.
 
BotFlex is an open source tool for bot detection and analysis. 
Its detection methodology is similar to BotHunter \textbf{reference}.
A bot carries several activities during its existence, which
when combined, comprise its lifecycle. These can be broadly 
categorized as Inbound scanning, Host exploit, Egg download, CnC
and Attack... 
     

\section{Design} 
\label{sec:design}

\subsection{To-Do} 
\label{sec:design_todo}

\section{Capabilities} 

\subsection{What BotFlex can do}
\begin{itemize}
\item Should be deployed close to home.
\end{itemize}

\subsection{What BotFlex cannot do}
\begin{itemize}
\item Won't do well in Bro Cluster setting because of the scan scripts.
Even otherwise, this tool is not meant for monitoring large traffic.
\end{itemize}

\subsection{To-Do}
\begin{itemize}
\item Replace the file reading script in bro.bif, util .h and util.cc
with the input framework that is expected to be released with Bro 2.1
\item If the input framework comes with Bro 2.1, arrange for the values
in config.txt to be changed at runtime.
\item Can we run BotFlex interactively?
\item Package it as a proper software.
\item Arrange for the values in config.txt to be set
via switches in command line.
\item Make skeleton attack.bro
\item Test communication with a Broccoli enabled remote server
\end{itemize}


\section{Extending BotFlex}
There is a standard way to interface scripts with the major events 
\textit{Scan}, \textit{Exploit}, \textit{Egg Download}, \textit{CnC} and \textit{Attack}. 
Similarly, a new phase like the ones just enumerated can be integrated
 into the \textit{Correlation} module in a distinct way. 
How to write the new script is a decision left to the coder. 
Nevertheless, there are some standard guidelines which can be utilized 
for writing scripts that adhere to the design philosophy of BotFlex.

As the core of BotFlex rests upon various events, 
one needs to decide forehand whether the addition
that has to be made justifies a full-fledge script
of its own, or a simple event in the main module of 
the phase under question. For example, a number of
services on the Internet distribute blacklists of
CnC servers based on information from their
globally deployed sensors (similar to honeypots). 
A very simple check in the context of the \textit{CnC}
module is to check if an IP within the monitored network
established communication with an IP from a CnC blacklist.
It would be superfluous to write a script \textit{cnc\textunderscore match.bro}
that has only a few lines of code to generate the event 
\textit{cnc-match}. On the other hand, spam generation is an 
important aspect of botnet \textit{Attack} phase, particularly 
in the context of spambots. Similarly, some botnets 
\cite{botnet-asprox}
 try to spread by injecting malicious links 
into random websites through SQL injection. It would be too
conservative to force SQL injection and spam detection
logic into the same script. By nature, these deserve
independent, stand-alone scripts of their own. This 
decision can be further facilitated by asking oneself 
the following questions: 
\begin{itemize}
\item The logic to be added is an independent phenomenon (case B)
or can be used to increase evidence of another independent 
phenomenon (case A).
\item Does it need a separate data structure to maintain
pertinent information, which can potentially be further extended.
\item Does its information need to go into data structure of another 
module?
\item Will you describe it as a complex event or a simple event? 
\end{itemize} 

Once that has been decided, we can look at how to handle 
the individual cases A and B.

\subsection{Case A: Simple Event}
Every phase of botnet life cycle is placed in a folder
named after the phase to which it corresponds. Within the
folder, the script with the name \textit{<dir\textunderscore name>.bro} 
accomodates simple events. The best way to understand the
process is to make use of an example.

Lets say we need to add a simple event \textit{e} to the phase
\textit{X} of botnet lifecycle. This can be done by following these
simple steps:
\begin{itemize}
\item Browse to the folder \textit{X} and edit \textit{X.bro}.
\item Make entries in the data structure \textit{XRecord}
that reflect information relevant to your logic which
you want to be logged and shared with other scripts if/when
required.
\item Update \textit{function getXRecord} to define intialization
of the newly added entries.
\item Add name of the simple \textit{event e} to the enum \textit{X\textunderscore tributary}
in \textit{export} block.
\item Add information about the recently added data items to
\textit{record Info} so that it can be logged.
\item If you want your additional data items to be available
in \textit{Correlation} module, you will need to edit signature of the 
main \textit{event X} in \textit{export} block. Moreover, modify the 
signature wherever \textit{event X} is handled. At the moment,
\textit{event X} is handled only in \textit{Correlation} module
and chances are that it will remain this way for some time.
  
\item Write logic that contributes to \textit{event X}.
Usually this involves handling other events and triggering
the \textit{evaluate} function when certain conditions are met.
As mentioned earlier, \textit{function evaluate} returns a
boolean value to indicate whether or not the major
\textit{event X} was triggered. 
\item This step pertains to when \textit{function evaluate}
yields `True'. Clean up the data relevant to the simple 
\textit{event e} from \textit{table\textunderscore X}. Additionally, delete the entry of 
\textit{event e} as defined in \textit{enum X\textunderscore tributary} from 
\textit{tb\textunderscore tributary} in \textit{table\textunderscore X}.  
\end{itemize}
If simple \textit{event e} uses some threshold values, the
following additional steps need to be taken to ensure they 
can be tweaked by editing the \textit{config.txt} file.
\begin{itemize}
\item Define the thresholds in \textit{export} block with some
default value.
\item Make entry in the file \textit{botflex/config.txt}
in this format:/sl{<X> <threshold\textunderscore name> <value>}
\item In \textit{event bro\textunderscore init}, arrange for the threshold value
to be pulled from the data structure that holds configurable
values as specified in \textit{botflex/config.txt}.
The typical syntax is:
\begin{verbatim}
if ( <threshold_name> in Config::table_config["X"] )
				{
				<threshold_name> = func(Config::table_config["X"]["threshold_name"]);
				}
\end{verbatim}

Note that if \textit{threshold\textunderscore name} is of type \textit{string}, 
\textit{func} in the above snippet might be skipped altogether. 
In most cases, the \textit{string} value needs to be converted into
proper format before handing it over from \textit{Config::table\textunderscore config}
to the script variable. The appropriate conversion function can be 
found in Bro's own list of utility functions in \textit{bro/src/bro.bif}
and \textit{bro/src/strings.bif}. Additionally, BotFlex also offers a few
useful functions of its own which can be found in 
\textit{botflex/utils/types.bro}.  				
\end{itemize}

\subsection{Case B: Complex Event}
Things are rather easy in this case as all you have to do
is to copy paste an existing script for a complex event
(\textit{attack/spam.bro} would be a good start)
and replace things with your own. The only consideration is
to decide how you want to interface your main event with
\textit{Correlation} module. For ease of discussion, lets assume 
that the complex event that has to be added to phase \textit{X} is called 
\textit{Y}. If \textit{Y} is so complex that it has a dozen data items 
that need to be reported to the \textit{Correlation} module.
It doesn't make sense to add these to the main \textit{event X}.
It not only makes the code look ugly ( imagine an event with a
signature that spans multiple lines ), but also it's difficult
to keep track of the order of signature data items wherever
\textit{event X} is handled. In this case, it's best to report 
\textit{event Y} to the \textit{Correlation} module as it is. 
Remember, we can tell the \textit{Correlation} module to tie different
events with the same botnet lifecycle phase. On the other hand,
if \textit{Y} is a complex event but not entirely independent, you
can report \textit{event Y} to \textit{X.bro}. This is the same as 
handling \textit{event Y} in \textit{X.bro}. Obviosuly, you will need to
make appropriate entries in \textit{X\textunderscore tributary}, \textit{XRecord} and 
\textit{getXRecord()}.

\subsection{Modifying Correlation Module}
This step is common to both cases A and B. Once the basic logic
has been implemented, you have to let \textit{Correlation} module know
so that it can be added to the evaluation process that decides
whether or not a given host is a bot.
The correlation module has two main parts. One that relates to 
making the `bot or not' decision, and the other is
responsible for maintaining a data structure that can 
be shared with a remote entity, lets say a correlation server.       

\subsubsection{Modifications related to `Decision'}
Lets continue with our previous example of adding event Y|y
to the correlation logic.
Whether \textit{event Y} piggybacks on
\textit{event X} or reports to correlation module independent
\textit{event Y}, a common set of actions is applicable. Cases
that are distinct to y or Y have been specifically identified:
\begin{itemize}
\item (For \textit{event Y}) Write event handler for \textit{event Y}.
\item Make additional entries in \textit{CorrelationRecord}.
\item Add initialization logic for these entries in  
\textit{get\textunderscore correlation\textunderscore record}
\item Update relevant entries in the event handler
and if certain conditions are met, update \textit{tb\textunderscore tributary} in
\textit{table\textunderscore tributary} and trigger the \textit{evaluate} function.
If the evaluation succeeds (i.e. returns True), clean up. 
\end{itemize}

\subsubsection{Modifications related to `Data'}
A number of data structures are used through out BotFlex
to hold pertinent data items. A common thread to all these
data structures (tables, to be more specific) is that they
expire after fixed time windows. This behavior is useful for
analyzing events in a temporal fashion. The flipside is that
we do not have persistent data anywhere. Imagine at some point
we want to share a summary of the activities of a host till the 
time it was declared as a bot with a remote party. One might 
suggest sharing the final log file with the interested party.
There are two main disadvantages to this approach:
\begin{itemize}
\item The log file can possibly contain a number of entries
for the same host. This burdens the receiving party with
the responsibility to use text processing tools to extract 
information of interest from the file.
\item If only a small piece of information is required by the 
remote party, it is not justifiable to send the entire log file
to R. 
  
For this reason, we maintain a data structure \textit{table\textunderscore bot}
that contains a summary of activities per host that was classified
as a bot. To add Y or y to this table, make appropriate entries in
\textit{BotRecord}, \textit{get\textunderscore bot\textunderscore record}, 
and other get\textunderscore * records if 
applicable. Wherever event related to Y or y is handled, make sure 
you update the corresponding values in \textit{table\textunderscore bot}.
\end{itemize}
      
\section{Installation}
BotFlex has not been packaged per se. 
This will happen soon, though. 
Until that happens, here is a quick guide that explains 
how to get BotFlex to work (Nix only :-)). 
The good news is that BotFlex does not have any
additional dependencies except for the ones it 
inherited from Bro. 

\begin{enumerate}
\item Download botflex from
\url{https://github.com/sheharbano/bsg/tree/master/botflex}
\item Download Bro source (which is a compressed folder of the form 
	bro.tar.gz). For details, please look at Appendix (this is not optional).
\item Extract to a place of your choice, but remember the path.
\item cd to the folder where Bro was extracted and modify bro source code as 
   described in botflex/bro\textunderscore modifications (5 minutes of copy pasting :-)). 
\item Assuming you are still in the directory where bro was extracted (bro),
   do the usual (i) ./configure (ii) make (iii) make install
\item Move botflex to /usr/local/bro/share/bro/site.
   Note: The above is the typical directory hierarchy of Bro unless you have	
         specifically changed the path during ./configure stage. If that's
	 the case, use the path you specified during configuration.
\item 5. For best results, considering modifying 
   /usr/local/bro/share/bro/site/botflex/config.txt
   according to your own environment. In particular, take time to
   specify local network. The default thresholds will be effective unless
   changed in the file mentioned above.
\item Add the following line to /usr/local/bro/share/bro/site/local.bro
   @load botflex/detection/correlation/correlation
\item We are all set now. For live traffic analysis, do 
   bro -i <interface> local.bro
   For analyzing a previously generated trace file, do
   bro -r <file.pcap> local.bro
\item Log files will be generated in your current directory. Bots are 
   listed in correlation.log (TO-DO). Other log filenames are self 
   explanatory.
\end{enumerate}
\section{Conclusion}

\appendix
\section{Appendix A: Downloading Bro}
The detailed quickstart guide can be found at 
\url{http://bro-ids.org/documentation/quickstart.html}
To cut a long story short, we further summarize Bro download here:
\begin{enumerate}
\item Download Bro: 
\begin{verbatim}
   git clone --recursive git://git.bro-ids.org/bro
\end{verbatim}

\item Get dependencies:
   (RPM/RedHat based Linux)
   \begin{verbatim}
   sudo yum install cmake make gcc gcc-c++ flex bison libpcap-devel 
   openssl-devel python-devel swig zlib-devel file-devel
   \end{verbatim}
   
   (DEB/Debian-based Linux)
   \begin{verbatim}
   sudo apt-get install cmake make gcc g++ flex bison libpcap-dev 
   libssl-dev python-dev swig zlib1g-dev libmagic-dev
   \end{verbatim}

   (Others)
   Check the above page. Haven't tested on other platforms.
\item Set path.
   (Bourne-shell syntax)
   \begin{verbatim}
   export PATH=/usr/local/bro/bin:\$PATH 
   \end{verbatim}
  
   (C-shell syntax) 
   \begin{verbatim}
   setenv PATH /usr/local/bro/bin:\$PATH
   \end{verbatim}
\appendix

\end{enumerate}

\bibliographystyle{acmtrans}
\bibliography{botflex-manual}


\end{document}


